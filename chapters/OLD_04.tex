\documentclass[output=paper]{langsci/langscibook} 
\author{Haim Dubossarsky\affiliation{University of Cambridge}\orcid{}} 
%\ORCIDs{}

%\markuptitle{A new theory of \textit{this} and \textit{that}}{A new theory of ``this'' and ``that''}

\title{Semantic change in the time of machine learning: Doing it right!}

\renewcommand{\lsCollectionPaperFooterTitle}{Semantic change in the time of machine learning: Doing it right!}

 

\abstract{
The chapter will discuss fundamentals of diachronic embedding methods for semantic change. It will also prove that the state-of-the-art models of lexical semantic change detection suffer from noise stemming from inaccurate representation and vector space alignment. First, word embedding is just a proxy for meaning. Comparing word embeddings at different time units can lead to findings that stem from noise. The chapter will then discuss the risks of using word embedding methods in regard to the field of semantic change detection.
 
The chapter will then elaborate on evaluation issues and propose statistically valid ways of evaluating proposed models. In particular it will introduce a principled way to simulate lexical semantic change and systematically control for possible biases using randomized control trials. It will then discuss issues with recently discovered laws in the area of diachronic semantic change analysis that are just artefact of frequency aspects in the data.
}

\begin{document}
\maketitle

%\section{Introduction} 
%Duis pulvinar lacus id gravida ornare. Phasellus eu mauris sed tortor maximus condimentum ultrices in leo. Donec non erat nec nulla ullamcorper ornare sed id ex. Integer risus mauris, aliquet vel aliquam sed, feugiat quis nisi. Suspendisse quis nunc a turpis porttitor mollis. In luctus nulla id nunc dapibus, id rhoncus lorem pretium. Nunc eget fringilla velit, semper commodo diam. Suspendisse odio odio, euismod ac ornare sed, tincidunt ac arcu. Pellentesque vitae fringilla orci. Donec faucibus metus dui, nec iaculis purus pellentesque sit amet. Sed fermentum lorem non augue cursus, eu accumsan risus ullamcorper. Suspendisse rhoncus magna vitae enim pellentesque, eget porttitor quam finibus. Nunc ultricies turpis at quam vehicula, at tempus justo molestie. Proin convallis augue ut turpis cursus rhoncus. Donec sed convallis justo. Sed sed massa pharetra ex aliquet eleifend. 
%\isi{finality} 


%Duis pulvinar lacus id gravida ornare. Phasellus eu mauris sed tortor maximus condimentum ultrices in leo. Donec non erat nec nulla ullamcorper ornare sed id ex. Integer risus mauris, aliquet vel aliquam sed, feugiat quis nisi. Suspendisse quis nunc a turpis porttitor mollis. In luctus nulla id nunc dapibus, id rhoncus lorem pretium. Nunc eget fringilla velit, semper commodo diam. Suspendisse odio odio, euismod ac ornare sed, tincidunt ac arcu. Pellentesque vitae fringilla orci. Donec faucibus metus dui, nec iaculis purus pellentesque sit amet. Sed fermentum lorem non augue cursus, eu accumsan risus ullamcorper. Suspendisse rhoncus magna vitae enim pellentesque, eget porttitor quam finibus. Nunc ultricies turpis at quam vehicula, at tempus justo molestie. Proin convallis augue ut turpis cursus rhoncus. Donec sed convallis justo. Sed sed massa pharetra ex aliquet eleifend. 
%\isi{finality} 
 
 
%\section*{Abbreviations}
%\begin{tabularx}{.45\textwidth}{lQ}
%... & \\
%... & \\
%\end{tabularx}
%\begin{tabularx}{.45\textwidth}{lQ}
%... & \\
%... & \\
%\end{tabularx}

%\section*{Acknowledgements}
%\citet{Nordhoff2018} is useful for compiling bibliographies.

{\sloppy\printbibliography[heading=subbibliography,notkeyword=this]}
\end{document}
