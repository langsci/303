\title{Computational approaches to semantic change}% \title{Computational Approaches to Lexical Semantic Change}
% \subtitle{Add subtitle here if exists}
%\author{Adam Jatowt\lastand Nina Tahmasebi\lastand Lars Borin\lastand Yang Xu\lastand Simon Hengchen} % order from the book proposal
\author{Nina Tahmasebi and Lars Borin and Adam Jatowt and Yang Xu and Simon Hengchen} % order from the ACL workshop
\renewcommand{\lsSeries}{lv}%use series acronym in lower case
\renewcommand{\lsSeriesNumber}{}
\renewcommand{\lsID}{303}
\BackBody{%
Semantic change -- how the meanings of words change over time -- has preoccupied scholars since well before modern linguistics emerged in the late 19th and early 20th century, ushering in a new methodological turn in the study of language change. Compared to changes in sound and grammar, semantic change is the least
understood. Ever since, the study of semantic change has progressed steadily, accumulating a vast store of knowledge for over a century, encompassing many languages and language families.

Historical linguists also early on realized the potential of computers as research tools, with papers at the very first international conferences in computational linguistics in the 1960s. Such computational studies still tended to be small-scale, method-oriented, and qualitative. However, recent years have witnessed a sea-change in this regard. Big-data empirical quantitative investigations are now coming to the forefront, enabled by enormous advances in storage capability and processing power. Diachronic corpora have grown beyond imagination, defying exploration by traditional manual qualitative methods, and language technology has become increasingly data-driven and semantics-oriented. These developments present a golden opportunity for the empirical study of semantic change over both long and short time spans.

A major challenge presently is to integrate the hard-earned
knowledge and expertise of traditional historical linguistics with
cutting-edge methodology explored primarily in computational linguistics.

The idea for the present volume came out of a concrete response to this challenge. 
The \emph{1st International Workshop on Computational Approaches to Historical Language Change} (LChange'19), at ACL 2019, brought together scholars from both fields.

This volume offers a survey of this exciting new direction in the study of semantic change, a discussion of the many remaining challenges that we face in pursuing it, and considerably updated and extended versions of a selection of the contributions to the LChange'19 workshop, addressing both more theoretical problems -- e.g., discovery of ``laws of semantic change'' -- and practical applications, such as information retrieval in longitudinal text archives.%
}
